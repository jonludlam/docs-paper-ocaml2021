\documentclass[a4paper,11pt,twocolumn]{scrartcl}

\usepackage[T1]{fontenc}
\usepackage[utf8]{inputenc}
\usepackage[english]{babel}

\usepackage{minted}

\usepackage[dvipsnames]{xcolor}
\usepackage{xspace}
\usepackage{fullpage}

\usepackage[scaled=0.8]{DejaVuSansMono} % A decent mono font

\usepackage{enumitem}
\setitemize{noitemsep,topsep=3pt,parsep=3pt,partopsep=3pt}
\setenumerate{noitemsep,topsep=3pt,parsep=3pt,partopsep=3pt}

\usepackage[bookmarks,colorlinks=true,citecolor=red]{hyperref}
\usepackage[noabbrev,capitalize]{cleveref}


\title{A Multiverse of Glorious Documentation}
\author{
  Lucas \textsc{Pluvinage}\\
  Tarides
  \and
  Jon \textsc{Ludlam}\\
  University of Cambridge
}
\date{}


\begin{document}
\maketitle

\begin{abstract}

This talk describes the process of generating documentation for every version
of every package that can be built from opam repository, and how it is presented
as a single coherent website that is continuously updated as new packages are
released and old packages are updated. The challenges of caching, of handling
different compiler versions, of incompatible libraries, and of OS requirements are 
all addressed. The process has been implemented as an OCurrent pipeline which
presents numerous benefits and also some challenges.

\section{The Goal}
The Opam repository contains many thousands of OCaml packages, many of which
contain embedded OCamldoc comments in their source code. These can be extracted along with the type
information to produce documentation of the module interfaces, something that
has been a core part of the OCaml ecosystem from the very early days, with
the tool ocamldoc maintained in the same repository as the compiler itself. This
includes hyperlinking between items, for example, if a type constructor is used
anywhere in the docs, it is possible to click on it to navigate to the declaration
of that type.

Ocamldoc focusses on producing what amounts to a nicely typeset and linked version
of mli files. However, as the language grew and libraries used more features, the documentation produced
for complex libraries by ocamldoc became less and less useful. Consequently odoc 
was created, which improved upon ocamldoc with a comprehensive model of the
language enabling expansion of module type expressions, functors, handling
of includes, handling of namespaces via 'wrapped' libraries, correct handling of functors and so on, producing
useful documentation even for libraries such as Jane Street's core. 

Whilst this is a huge improvement on ocamldoc, it relies on type information provided
by the compiler in the form of the cmt or cmti files. As such, the main drivers of
odoc, odig and dune, could only operate on a set of packages that are co-installable.
Our goal was to drop this requirement, and thus extend the documentation process such that it is possible to use odoc to produce a single tree of
package documentation that covers every single version of every single package found in
the opam repository.

\section{Dependency Universes}
At first glance it is not unreasonable to suggest that the solution to this problem is simply
to sequentially install a particular version of a particular package, obtain its docs via odoc, arranging
for the results to be put in a suitably named directory -- perhaps "packages/<package name>/<package version/" --
and repeat for every other version of every other package. Why would this not work?
The big problem is that the types in a particular version of a package, and therefore the
documentation for a particular version of a package, may depend upon
the types in dependent packages not just the package itself. As a simple example, a
package may expose a Set or Map using the functors in the standard library, and the
signatures of these are dependent on the version of the compiler rather than the
version of the package. As a simple consequence of this, the naive solution to this
problem would end up with inconsistent documentation with many broken hyperlinks.

It is therefore vitally important that we keep track of the exact
versions of all the dependency packages used to build each subsequent package. This is
termed the 'dependency universe', and is the transitive closure of the set of the versions
of opam packages that opam declares is required.



In order to 


For example, the package `astring.0.8.5`
declares dependencies on ocaml, ocamlbuild, ocamlfind and topkg with various constraints.
These are solved using the 0install solver, and we might find a resulting dependency universe consisting of

```
conf-m4.1
ocaml.4.11.1
ocamlbuild.0.14.0
ocamlfind.1.8.1
topkg.1.0.3
```

where conf-m4 has been added as a transitive dependency. A simple hash, in this case md5, is
then used to provide an id for this dependency universe, an


\end{abstract}

\section{The model layer}

\paragraph{Module expansion}

\end{document}

%%% Local Variables:
%%% mode: latex
%%% TeX-master: t
%%% TeX-command-extra-options: "-shell-escape"
%%% End:
